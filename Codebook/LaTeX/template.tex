%% ---------------------------------------------------
%% Notice that we use the "report" class instead of "article"
%% ---------------------------------------------------
\documentclass{article}

\title{MSDS 6371\\
 TEMPLATE}
\author{Travis Daun}
\date{\today}

%% ---------------------------------------------------
%% 634format specifies the format of our reports
%% ---------------------------------------------------
\usepackage{634format}

%% ---------------------------------------------------
%% enumerate 
%% ---------------------------------------------------
\usepackage{enumerate}

%% ---------------------------------------------------
%% listings is used for including our source code in reports
%% textcomp provides additional symbols
%% ---------------------------------------------------
\usepackage{listings}
\usepackage{textcomp}
\usepackage{pdfpages}

%% ---------------------------------------------------
%% Packages for math environments
%% ---------------------------------------------------
\usepackage{amsmath}

%% ---------------------------------------------------
%% Packages for URLs and hotlinks in the table of contents
%% and symbolic cross references using \ref
%% ---------------------------------------------------
\usepackage{hyperref}

\usepackage[most]{tcolorbox}
\usepackage[utf8]{inputenc}
\usepackage{color,soul}
\usepackage{pgfplots}

%% ---------------------------------------------------
%% Packages for using HOL-generated macros and displays
%% ---------------------------------------------------
\usepackage{holtex}
\usepackage{holtexbasic}
% =====================================================================
%
% Macros for typesetting the HOL system manual
%
% =====================================================================

% ---------------------------------------------------------------------
% Abbreviations for words and phrases
% ---------------------------------------------------------------------

\newcommand\TUTORIAL{{\footnotesize\sl TUTORIAL}}
\newcommand\DESCRIPTION{{\footnotesize\sl DESCRIPTION}}
\newcommand\REFERENCE{{\footnotesize\sl REFERENCE}}
\newcommand\LOGIC{{\footnotesize\sl LOGIC}}
\newcommand\LIBRARIES{{\footnotesize\sl LIBRARIES}}
\usepackage{textcomp}

\newcommand{\bs}{\texttt{\char'134}} % backslash
\newcommand{\lb}{\texttt{\char'173}} % left brace
\newcommand{\rb}{\texttt{\char'175}} % right brace
\newcommand{\td}{\texttt{\char'176}} % tilde
\newcommand{\lt}{\texttt{\char'74}} % less than
\newcommand{\gt}{\texttt{\char'76}} % greater than
\newcommand{\dol}{\texttt{\char'44}} % dollar
\newcommand{\pipe}{\texttt{\char'174}}
\newcommand{\apost}{\texttt{\textquotesingle}}
% double back quotes ``
\newcommand{\dq}{\texttt{\char'140\char'140}}
%These macros were included by slind:

\newcommand{\holquote}[1]{\dq#1\dq}

\def\HOL{\textsc{Hol}}
\def\holn{\HOL}  % i.e. hol n(inety-eight), no digits in
                 % macro names is a bit of a pain; deciding to do away
                 % with hol98 nomenclature means that we just want to
                 % write HOL for hol98.
\def\holnversion{Kananaskis-11}
\def\holnsversion{Kananaskis~11} % version with space rather than hyphen
\def\LCF{{\small LCF}}
\def\LCFLSM{{\small LCF{\kern-.2em}{\normalsize\_}{\kern0.1em}LSM}}
\def\PPL{{\small PP}{\kern-.095em}$\lambda$}
\def\PPLAMBDA{{\small PPLAMBDA}}
\def\ML{{\small ML}}
\def\holmake{\texttt{Holmake}}

\newcommand\ie{\mbox{\textit{i{.}e{.}}}}
\newcommand\eg{\mbox{\textit{e{.}g{.}}}}
\newcommand\viz{\mbox{viz{.}}}
\newcommand\adhoc{\mbox{\it ad hoc}}
\newcommand\etal{{\it et al.\/}}
% NOTE: \etc produces wrong spacing if used between sentences, that is
% like here \etc End such sentences with non-macro etc.
\newcommand\etc{\mbox{\textit{etc{.}}}}

% ---------------------------------------------------------------------
% Simple abbreviations and macros for mathematical typesetting
% ---------------------------------------------------------------------

\newcommand\fun{{\to}}
\newcommand\prd{{\times}}

\newcommand\conj{\ \wedge\ }
\newcommand\disj{\ \vee\ }
\newcommand\imp{ \Rightarrow }
\newcommand\eqv{\ \equiv\ }
\newcommand\cond{\rightarrow}
\newcommand\vbar{\mid}
\newcommand\turn{\ \vdash\ } % FIXME: "\ " results in extra space
\newcommand\hilbert{\varepsilon}
\newcommand\eqdef{\ \equiv\ }

\newcommand\natnums{\mbox{${\sf N}\!\!\!\!{\sf N}$}}
\newcommand\bools{\mbox{${\sf T}\!\!\!\!{\sf T}$}}

\newcommand\p{$\prime$}
\newcommand\f{$\forall$\ }
\newcommand\e{$\exists$\ }

\newcommand\orr{$\vee$\ }
\newcommand\negg{$\neg$\ }

\newcommand\arrr{$\rightarrow$}
\newcommand\hex{$\sharp $}

\newcommand{\uquant}[1]{\forall #1.\ }
\newcommand{\equant}[1]{\exists #1.\ }
\newcommand{\hquant}[1]{\hilbert #1.\ }
\newcommand{\iquant}[1]{\exists ! #1.\ }
\newcommand{\lquant}[1]{\lambda #1.\ }

\newcommand{\leave}[1]{\\[#1]\noindent}
\newcommand\entails{\mbox{\rule{.3mm}{4mm}\rule[2mm]{.2in}{.3mm}}}

% ---------------------------------------------------------------------
% Font-changing commands
% ---------------------------------------------------------------------

\newcommand{\theory}[1]{\hbox{{\small\tt #1}}}
\newcommand{\theoryimp}[1]{\texttt{#1}}

\newcommand{\con}[1]{{\sf #1}}
\newcommand{\rul}[1]{{\tt #1}}
\newcommand{\ty}[1]{\textsl{#1}}

\newcommand{\ml}[1]{\mbox{{\def\_{\char'137}\texttt{#1}}}}
\newcommand{\holtxt}[1]{\ml{#1}}
\newcommand\ms{\tt}
\newcommand{\s}[1]{{\small #1}}

\newcommand{\pin}[1]{{\bf #1}}
% FIXME: for multichar symbols \mathit should be used.
\def\m#1{\mbox{\normalsize$#1$}}

% ---------------------------------------------------------------------
% Abbreviations for particular mathematical constants etc.
% ---------------------------------------------------------------------

\newcommand\T{\con{T}}
\newcommand\F{\con{F}}
\newcommand\OneOne{\con{One\_One}}
\newcommand\OntoSubset{\con{Onto\_Subset}}
\newcommand\Onto{\con{Onto}}
\newcommand\TyDef{\con{Type\_Definition}}
\newcommand\Inv{\con{Inv}}
\newcommand\com{\con{o}}
\newcommand\Id{\con{I}}
\newcommand\MkPair{\con{Mk\_Pair}}
\newcommand\IsPair{\con{Is\_Pair}}
\newcommand\Fst{\con{Fst}}
\newcommand\Snd{\con{Snd}}
\newcommand\Suc{\con{Suc}}
\newcommand\Nil{\con{Nil}}
\newcommand\Cons{\con{Cons}}
\newcommand\Hd{\con{Hd}}
\newcommand\Tl{\con{Tl}}
\newcommand\Null{\con{Null}}
\newcommand\ListPrimRec{\con{List\_Prim\_Rec}}


\newcommand\SimpRec{\con{Simp\_Rec}}
\newcommand\SimpRecRel{\con{Simp\_Rec\_Rel}}
\newcommand\SimpRecFun{\con{Simp\_Rec\_Fun}}
\newcommand\PrimRec{\con{Prim\_Rec}}
\newcommand\PrimRecRel{\con{Prim\_Rec\_Rel}}
\newcommand\PrimRecFun{\con{Prim\_Rec\_Fun}}

\newcommand\bool{\ty{bool}}
\newcommand\num{\ty{num}}
\newcommand\ind{\ty{ind}}
\newcommand\lst{\ty{list}}

% ---------------------------------------------------------------------
% \minipagewidth = \textwidth minus 1.02 em
% ---------------------------------------------------------------------

\newlength{\minipagewidth}
\setlength{\minipagewidth}{\textwidth}
\addtolength{\minipagewidth}{-1.02em}

% ---------------------------------------------------------------------
% Environment for the items on the title page of a case study
% ---------------------------------------------------------------------

\newenvironment{inset}[1]{\noindent{\large\bf #1}\begin{list}%
{}{\setlength{\leftmargin}{\parindent}%
\setlength{\topsep}{-.1in}}\item }{\end{list}\vskip .4in}

% ---------------------------------------------------------------------
% Macros for little HOL sessions displayed in boxes.
%
% Usage: (1) \setcounter{sessioncount}{1} resets the session counter
%
%        (2) \begin{session}\begin{verbatim}
%             .
%              < lines from hol session >
%             .
%            \end{verbatim}\end{session}
%
%            typesets the session in a numbered box.
% ---------------------------------------------------------------------

\newlength{\hsbw}
\setlength{\hsbw}{\textwidth}
\addtolength{\hsbw}{-\arrayrulewidth}
\addtolength{\hsbw}{-\tabcolsep}
\newcommand\HOLSpacing{13pt}

\newcounter{sessioncount}
\setcounter{sessioncount}{0}

\newenvironment{session}{\begin{flushleft}
 \refstepcounter{sessioncount}
 \begin{tabular}{@{}|c@{}|@{}}\hline
 \begin{minipage}[b]{\hsbw}
 \vspace*{-.5pt}
 \begin{flushright}
 \rule{0.01in}{.15in}\rule{0.3in}{0.01in}\hspace{-0.35in}
 \raisebox{0.04in}{\makebox[0.3in][c]{\footnotesize\sl \thesessioncount}}
 \end{flushright}
 \vspace*{-.55in}
 \begingroup\small\baselineskip\HOLSpacing}{\endgroup\end{minipage}\\ \hline
 \end{tabular}
 \end{flushleft}}

% ---------------------------------------------------------------------
% Macro for boxed ML functions, etc.
%
% Usage: (1) \begin{holboxed}\begin{verbatim}
%               .
%               < lines giving names and types of mk functions >
%               .
%            \end{verbatim}\end{holboxed}
%
%            typesets the given lines in a box.
%
%            Conventions: lines are left-aligned under the "g" of begin,
%            and used to highlight primary reference for the ml function(s)
%            that appear in the box.
% ---------------------------------------------------------------------

\newenvironment{holboxed}{\begin{flushleft}
  \begin{tabular}{@{}|c@{}|@{}}\hline
  \begin{minipage}[b]{\hsbw}
% \vspace*{-.55in}
  \vspace*{.06in}
  \begingroup\small\baselineskip\HOLSpacing}{\endgroup\end{minipage}\\ \hline
  \end{tabular}
  \end{flushleft}}

% ---------------------------------------------------------------------
% Macro for unboxed ML functions, etc.
%
% Usage: (1) \begin{hol}\begin{verbatim}
%               .
%               < lines giving names and types of mk functions >
%               .
%            \end{verbatim}\end{hol}
%
%            typesets the given lines exactly like {boxed}, except there's
%            no box.
%
%            Conventions: lines are left-aligned under the "g" of begin,
%            and used to display ML code in verbatim, left aligned.
% ---------------------------------------------------------------------

\newenvironment{hol}{\begin{flushleft}
 \begin{tabular}{c@{}@{}}
 \begin{minipage}[b]{\hsbw}
% \vspace*{-.55in}
 \vspace*{.06in}
 \begingroup\small\baselineskip\HOLSpacing}{\endgroup\end{minipage}\\
 \end{tabular}
 \end{flushleft}}

% ---------------------------------------------------------------------
% Emphatic brackets
% ---------------------------------------------------------------------

\newcommand\leb{\lbrack\!\lbrack}
\newcommand\reb{\rbrack\!\rbrack}


% ---------------------------------------------------------------------
% Quotations
% ---------------------------------------------------------------------


%These macros were included by ap; they are used in Chapters 9 and 10
%of the HOL DESCRIPTION

\newcommand{\inds}%standard infinite set
 {\mbox{\rm I}}

\newcommand{\ch}%standard choice function
 {\mbox{\rm ch}}

\newcommand{\den}[1]%denotational brackets
 {[\![#1]\!]}

\newcommand{\two}%standard 2-element set
 {\mbox{\rm 2}}

\pgfmathdeclarefunction{gauss}{3}{%
  \pgfmathparse{1/(#3*sqrt(2*pi))*exp(-((#1-#2)^2)/(2*#3^2))}%
}
\begin{document}

%% --------------------------------------------------- the listings
%% parameter "language" is set to "ML"
%% ---------------------------------------------------
\lstset{language=R,
    basicstyle=\small\ttfamily,
    stringstyle=\color{DarkGreen},
    otherkeywords={0,1,2,3,4,5,6,7,8,9},
    morekeywords={TRUE,FALSE},
    deletekeywords={data,frame,length,as,character},
    keywordstyle=\color{blue},
    commentstyle=\color{DarkGreen},
}
\lstset{language=SAS, 
  breaklines=true,  
  basicstyle=\ttfamily\bfseries,
  columns=fixed,
  keepspaces=true,
  identifierstyle=\color{blue}\ttfamily,
  keywordstyle=\color{cyan}\ttfamily,
  stringstyle=\color{purple}\ttfamily,
  commentstyle=\color{green}\ttfamily,
  } 

\maketitle{}



%% -----------------------------------------------------------
%% Warning: don't copy and paste chapter, section, and subsection
%% commands with their labels.  This degrades/confuses cross
%% references and URL hotlinks within your document!
%% Use C-c C-s to introduce new sections and labels.
%% If automatic labels don't work right, execute M-x reftex-mode
%% (usually twice) to toggle reftex (automatic labeling) off and
%% on.
%% -----------------------------------------------------------

%\begin{tcolorbox}[colback=red!5!white,colframe=red!75!black,title=Self Grading: Overall]
%\textbf{\%}\\
%\begin{tabular}{|c|c|c|}
%\hline
%Question & Points & Points\\
%& Earned & Available\\
%\hline
%Turning in Assignment & $ $ & $ $\\
%\hline
%Question 1 & $ $ & $  $\\
%\hline
%Question 2 & $ $ & $  $\\
%\hline
%Question 3 & $ $ & $  $\\
%\hline
%Question 4 & $ $ & $  $\\
%\hline
%Question 5 & $ $ & $  $\\
%\hline
%Question Bonus & $ $ & $  $\\
%\hline
%Total & $ $ & $ $\\
%\hline
%\hline
%\end{tabular}\\
%See individual rubics at end of each answer for details/comments.
%\end{tcolorbox}

\begin{itemize}
\item[1.] 
\item[2.]
\begin{itemize}
\item[a.]
\item[b.]
\end{itemize}
\item[3.] 
\end{itemize}
\pagebreak

\begin{tcolorbox}[title=Answer 2]
\begin{tikzpicture}
\begin{axis}[
  no markers, 
  domain=0:6, 
  samples=100,
  ymin=0,
  axis lines*=left, 
  xlabel=$x$,
  every axis y label/.style={at=(current axis.above origin),anchor=south},
  every axis x label/.style={at=(current axis.right of origin),anchor=west},
  height=5cm, 
  width=12cm,
  xtick=\empty, 
  ytick=\empty,
  enlargelimits=false, 
  clip=false, 
  axis on top,
  grid = major,
  hide y axis
  ]

 \addplot [very thick,cyan!50!black] {gauss(x, 3, 1)};

\pgfmathsetmacro\valueA{gauss(1,3,1)}
\pgfmathsetmacro\valueB{gauss(2,3,1)}
\pgfmathsetmacro\valueC{gauss(3,3,1)}
\pgfmathsetmacro\valueD{gauss(0,3,1)}
\draw [very thick,black] (axis cs:5,0) -- (axis cs:5,\valueA);
    
\draw [gray] (axis cs:3,0) -- (axis cs:3,\valueC);
%\draw [yshift=-0.2cm, latex-|](axis cs:0, 0) -- node [] {} (axis cs:1, 0);
\draw [yshift=-0.2cm, |-latex](axis cs:5, 0) -- node [] {} (axis cs:6, 0);
\draw [yshift=-0.8cm](axis cs:0, 0) -- node [] {} (axis cs:6, 0);
\draw [yshift=-1cm,very thick,black]    (axis cs:5,0) -- (axis cs:5,\valueA);
\node[yshift = 2cm, above right] at (axis cs:4, 0)  {$\overline{\rm X}_{IncE16} - \overline{\rm X}_{IncE12}$};

%\draw [yshift=.4cm, latex-latex](axis cs:5, 0) -- node [fill=white] {$0.05$} (axis cs:6, 0);
\node[yshift=-1.1cm, below] at (axis cs:5, 0)  {$1.648$};
\node[yshift=-1.5cm, below] at (axis cs:5, 0)  {$t_{0.05,df=473}$};
\node[yshift=-0.2cm, below] at (axis cs:5.5, 0)  {shade}; 
\node[below] at (axis cs:3, 0)  {$0$}; 
\draw[yshift = 1cm, above right, -latex] (axis cs:5.5, 0) -- node []  {$0.05$} (axis cs:5.5,-.1);

\end{axis}
\end{tikzpicture}

	\end{tcolorbox}
\pagebreak

%\begin{tcolorbox}[colback=red!5!white,colframe=red!75!black,title=Self Grading: Question 1]
%\begin{tabular}{|c c c|c|l}
%\hline
%1.a & Normality & & $+1.5$ & $-1.5$ Plotted total dataset\\
%&&&&vice each class\\
%& Equal SD & & $+1.5$ & $-1.5$ Plotted total dataset\\
%&&&&vice each class\\
%& Independence & & $+3$\\
%& Decision & & $+1$\\
%\hline
%1.b & R && $+8$ & $-2$ Plotted total dataset\\
%&&&&vice each class\\
%\hline
%1.c & Problem & &$+1$\\
%& Assumption & &$+1$\\
%&& Hyp & $+1$\\
%& & Draw & $+1$\\
%& & Critical Value & $+1$\\
%& & Test Statistic & $+1$\\
%& & p-value & $+1$\\
%& Decision & Conclusion & $+1$\\
%& & Confidence Interval & $+1$\\
%& & Scope & $+1$\\
%\hline
%Total & & & $25/30$\\
%\hline
%\hline
%\end{tabular}
%\end{tcolorbox}
\pagebreak


%% ------------------------------------------
%% this restarts the section numbering and
%% changes chapter numbering to letters starting
%% with A
%% ------------------------------------------
\appendix{} 


\section{Source Code for Problem 2}
\label{Problem2}

\textbf{The following code is for \emph{Logging.sas}}
%\lstinputlisting[language=SAS]{../code/Logging.sas}

\noindent \textbf{The following code is for \emph{Logging.r}}
%\lstinputlisting[language=r]{../code/Logging.r}



\end{document}