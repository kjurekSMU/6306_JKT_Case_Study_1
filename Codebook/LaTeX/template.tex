%% ---------------------------------------------------
%% Notice that we use the "report" class instead of "article"
%% ---------------------------------------------------
\documentclass{article}

\title{MSDS 6371\\
 TEMPLATE}
\author{Travis Daun}
\date{\today}

%% ---------------------------------------------------
%% 634format specifies the format of our reports
%% ---------------------------------------------------
\usepackage{634format}

%% ---------------------------------------------------
%% enumerate 
%% ---------------------------------------------------
\usepackage{enumerate}

%% ---------------------------------------------------
%% listings is used for including our source code in reports
%% textcomp provides additional symbols
%% ---------------------------------------------------
\usepackage{listings}
\usepackage{textcomp}
\usepackage{pdfpages}

%% ---------------------------------------------------
%% Packages for math environments
%% ---------------------------------------------------
\usepackage{amsmath}

%% ---------------------------------------------------
%% Packages for URLs and hotlinks in the table of contents
%% and symbolic cross references using \ref
%% ---------------------------------------------------
\usepackage{hyperref}

\usepackage[most]{tcolorbox}
\usepackage[utf8]{inputenc}
\usepackage{color,soul}
\usepackage{pgfplots}

%% ---------------------------------------------------
%% Packages for using HOL-generated macros and displays
%% ---------------------------------------------------
\usepackage{holtex}
\usepackage{holtexbasic}
\input{commands}
\pgfmathdeclarefunction{gauss}{3}{%
  \pgfmathparse{1/(#3*sqrt(2*pi))*exp(-((#1-#2)^2)/(2*#3^2))}%
}
\begin{document}

%% --------------------------------------------------- the listings
%% parameter "language" is set to "ML"
%% ---------------------------------------------------
\lstset{language=R,
    basicstyle=\small\ttfamily,
    stringstyle=\color{DarkGreen},
    otherkeywords={0,1,2,3,4,5,6,7,8,9},
    morekeywords={TRUE,FALSE},
    deletekeywords={data,frame,length,as,character},
    keywordstyle=\color{blue},
    commentstyle=\color{DarkGreen},
}
\lstset{language=SAS, 
  breaklines=true,  
  basicstyle=\ttfamily\bfseries,
  columns=fixed,
  keepspaces=true,
  identifierstyle=\color{blue}\ttfamily,
  keywordstyle=\color{cyan}\ttfamily,
  stringstyle=\color{purple}\ttfamily,
  commentstyle=\color{green}\ttfamily,
  } 

\maketitle{}



%% -----------------------------------------------------------
%% Warning: don't copy and paste chapter, section, and subsection
%% commands with their labels.  This degrades/confuses cross
%% references and URL hotlinks within your document!
%% Use C-c C-s to introduce new sections and labels.
%% If automatic labels don't work right, execute M-x reftex-mode
%% (usually twice) to toggle reftex (automatic labeling) off and
%% on.
%% -----------------------------------------------------------

%\begin{tcolorbox}[colback=red!5!white,colframe=red!75!black,title=Self Grading: Overall]
%\textbf{\%}\\
%\begin{tabular}{|c|c|c|}
%\hline
%Question & Points & Points\\
%& Earned & Available\\
%\hline
%Turning in Assignment & $ $ & $ $\\
%\hline
%Question 1 & $ $ & $  $\\
%\hline
%Question 2 & $ $ & $  $\\
%\hline
%Question 3 & $ $ & $  $\\
%\hline
%Question 4 & $ $ & $  $\\
%\hline
%Question 5 & $ $ & $  $\\
%\hline
%Question Bonus & $ $ & $  $\\
%\hline
%Total & $ $ & $ $\\
%\hline
%\hline
%\end{tabular}\\
%See individual rubics at end of each answer for details/comments.
%\end{tcolorbox}

\begin{itemize}
\item[1.] 
\item[2.]
\begin{itemize}
\item[a.]
\item[b.]
\end{itemize}
\item[3.] 
\end{itemize}
\pagebreak

\begin{tcolorbox}[title=Answer 2]
\begin{tikzpicture}
\begin{axis}[
  no markers, 
  domain=0:6, 
  samples=100,
  ymin=0,
  axis lines*=left, 
  xlabel=$x$,
  every axis y label/.style={at=(current axis.above origin),anchor=south},
  every axis x label/.style={at=(current axis.right of origin),anchor=west},
  height=5cm, 
  width=12cm,
  xtick=\empty, 
  ytick=\empty,
  enlargelimits=false, 
  clip=false, 
  axis on top,
  grid = major,
  hide y axis
  ]

 \addplot [very thick,cyan!50!black] {gauss(x, 3, 1)};

\pgfmathsetmacro\valueA{gauss(1,3,1)}
\pgfmathsetmacro\valueB{gauss(2,3,1)}
\pgfmathsetmacro\valueC{gauss(3,3,1)}
\pgfmathsetmacro\valueD{gauss(0,3,1)}
\draw [very thick,black] (axis cs:5,0) -- (axis cs:5,\valueA);
    
\draw [gray] (axis cs:3,0) -- (axis cs:3,\valueC);
%\draw [yshift=-0.2cm, latex-|](axis cs:0, 0) -- node [] {} (axis cs:1, 0);
\draw [yshift=-0.2cm, |-latex](axis cs:5, 0) -- node [] {} (axis cs:6, 0);
\draw [yshift=-0.8cm](axis cs:0, 0) -- node [] {} (axis cs:6, 0);
\draw [yshift=-1cm,very thick,black]    (axis cs:5,0) -- (axis cs:5,\valueA);
\node[yshift = 2cm, above right] at (axis cs:4, 0)  {$\overline{\rm X}_{IncE16} - \overline{\rm X}_{IncE12}$};

%\draw [yshift=.4cm, latex-latex](axis cs:5, 0) -- node [fill=white] {$0.05$} (axis cs:6, 0);
\node[yshift=-1.1cm, below] at (axis cs:5, 0)  {$1.648$};
\node[yshift=-1.5cm, below] at (axis cs:5, 0)  {$t_{0.05,df=473}$};
\node[yshift=-0.2cm, below] at (axis cs:5.5, 0)  {shade}; 
\node[below] at (axis cs:3, 0)  {$0$}; 
\draw[yshift = 1cm, above right, -latex] (axis cs:5.5, 0) -- node []  {$0.05$} (axis cs:5.5,-.1);

\end{axis}
\end{tikzpicture}

	\end{tcolorbox}
\pagebreak

%\begin{tcolorbox}[colback=red!5!white,colframe=red!75!black,title=Self Grading: Question 1]
%\begin{tabular}{|c c c|c|l}
%\hline
%1.a & Normality & & $+1.5$ & $-1.5$ Plotted total dataset\\
%&&&&vice each class\\
%& Equal SD & & $+1.5$ & $-1.5$ Plotted total dataset\\
%&&&&vice each class\\
%& Independence & & $+3$\\
%& Decision & & $+1$\\
%\hline
%1.b & R && $+8$ & $-2$ Plotted total dataset\\
%&&&&vice each class\\
%\hline
%1.c & Problem & &$+1$\\
%& Assumption & &$+1$\\
%&& Hyp & $+1$\\
%& & Draw & $+1$\\
%& & Critical Value & $+1$\\
%& & Test Statistic & $+1$\\
%& & p-value & $+1$\\
%& Decision & Conclusion & $+1$\\
%& & Confidence Interval & $+1$\\
%& & Scope & $+1$\\
%\hline
%Total & & & $25/30$\\
%\hline
%\hline
%\end{tabular}
%\end{tcolorbox}
\pagebreak


%% ------------------------------------------
%% this restarts the section numbering and
%% changes chapter numbering to letters starting
%% with A
%% ------------------------------------------
\appendix{} 


\section{Source Code for Problem 2}
\label{Problem2}

\textbf{The following code is for \emph{Logging.sas}}
%\lstinputlisting[language=SAS]{../code/Logging.sas}

\noindent \textbf{The following code is for \emph{Logging.r}}
%\lstinputlisting[language=r]{../code/Logging.r}



\end{document}